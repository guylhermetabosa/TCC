\include{fixos/pacoteseclass}
\setlrmarginsandblock{3cm}{2cm}{*}
\setulmarginsandblock{3cm}{2cm}{*}
\checkandfixthelayout

\usepackage{float}
%\usepackage[a4paper,bottom=2cm,top=3cm,left=3cm,right=2cm]{geometry}


% Informações de dados para CAPA e FOLHA DE ROSTO
\titulo{\uppercase{Implantação de uma ferramenta de integração contínua em um Núcleo de Práticas em informática: Relato de Experiênca}}
\autor{Guylherme Tabosa Cabral}
\local{Quixadá, Ceará}
\data{2014}
\orientador{Prof Msc. Carlos Diego Andrade de Almeida}

% TODO: Fazer para SI, CC, ES
\instituicao{%
Universidade Federal do Ceará
\par
Campus Quixadá
\par
Curso de Engenharia de Software
}
\tipotrabalho{Trabalho de Conclusão de Curso (Monografia)}
\preambulo{Trabalho de Conclusão de Curso submetido à Coordenação do Curso de Engenharia de Software do Campus Quixadá da Universidade Federal do Ceará, como requisito parcial para obtenção do Título de Bacharel em Engenharia de Software.}


\begin{document}
\frenchspacing 

% ----------------------------------------------------------
% ELEMENTOS PRÉ-TEXTUAIS
% ----------------------------------------------------------
\pretextual
% Capa
\imprimircapa
% Folha de rosto (* indica que haverá a ficha bibliográfica)
%\imprimirfolhaderosto*

% Ficha Bibliográfica
%\include{fixos/fichabibliografica}

% Errata
%\include{editaveis/errata}

% Folha de Aprovação
%% ---
% Inserir folha de aprovação
% ---

% Isto é um exemplo de Folha de aprovação, elemento obrigatório da NBR
% 14724/2011 (seção 4.2.1.3). Você pode utilizar este modelo até a aprovação
% do trabalho. Após isso, substitua todo o conteúdo deste arquivo por uma
% imagem da página assinada pela banca com o comando abaixo:
%
% \includepdf{folhadeaprovacao_final.pdf}
%
\begin{folhadeaprovacao}

  \begin{center}
    {\ABNTEXchapterfont\bfseries\Large\imprimirautor}
    \vspace{1cm}

    \begin{center}
      \ABNTEXchapterfont\bfseries\Large\imprimirtitulo
    \end{center}

    \vspace{2cm}
    \begin{minipage}{\textwidth}
        \imprimirpreambulo
        \\ \\ \\
        Aprovada em: \_\_/\_\_/\_\_\_\_
    \end{minipage}%
     
    \vspace{2cm}
	\textbf{BANCA EXAMINADORA}
   \end{center}
	

   \assinatura{\imprimirorientador \space (Orientador) \\ Universidade Federal do Ceará (UFC)}
 
   \assinatura{Prof Msc. Camilo Camilo Almendra\\ Universidade Federal do Ceará (UFC)}
   
   \assinatura{Prof Dr. Lincoln Souza Rocha \space (Membro) \\ Universidade Federal do Ceará (UFC)}
   %\assinatura{\textbf{Professor} \\ Convidado 2}
   %\assinatura{\textbf{Professor} \\ Convidado 3}
      
%   \begin{center}
%    \vspace*{0.5cm}
%    {\large\imprimirlocal}
%    \par
%    {\large\imprimirdata}
%    \vspace*{1cm}
%  \end{center}
  
\end{folhadeaprovacao}
% ---

%\imprimirfolhadeaprovacao

% Dedicatória
%% ---
% Dedicatória
% ---
\begin{dedicatoria}
   \vspace*{\fill}
   	\begin{flushright}
   \noindent
    Dedico este trabalho a minha família principalmente meus avós, minha mãe, meu irmão e meu pai.
   	\end{flushright}
\end{dedicatoria}
% ---

% Agradecimentos
%% ---
% Agradecimentos
% ---
\begin{agradecimentos}
	 Agradeço primeiramente a Deus por me dar sabedoria e vontade para concluir este trabalho.A minha mãe Rejane, pelo exemplo de mulher e por sempre olhar por mim quando não estive perto durante esses anos de universidade. A meus avós Rita e Nonato por todo o apoio, amor e por serem meu maior exemplo de vida. A meu irmão Felype por sempre partilhar comigo momentos bons e ruins e cuidar de mim quando precisei. A meu pai João por ensinar que o trabalho sempre gera resultados, que o esforço é seu maior aliado para o sucesso. A minha família pela ajuda e cuidado. A minha namorada Mikaely por ter me aturado nos momentos de confusão, medo, raiva angústia e outros mais, sua presença ao meu lado me fez mais forte e focado nos objetivos. A meus amigos de faculdade, por todos esses anos de companheirismo, felicidades ajudas. A meu orientador Carlos Diego por me guiar na obtenção do melhor resultado possível e disponibilidade e interesse neste projeto. A UFC e seus funcionários por fornecer as ferramentas para que pudesse evoluir meus conhecimentos. Obrigado a todos.
\end{agradecimentos}
% ---

% Epígrafe
%% ---
% Epígrafe
% ---
\begin{epigrafe}
    \vspace*{\fill}
	\begin{flushright}
		\textit{``Observo a mim mesmo em silêncio, \\
		porque é nele onde mais e melhor se diz, \\
		Me  ensino a ser mais tolerante, não julgar ninguém \\
		E com isso ser mais feliz.\\
		''}
	\end{flushright}
\end{epigrafe}
% ---

% RESUMOS
%% resumo em português
\setlength{\absparsep}{18pt} % ajusta o espaçamento dos parágrafos do resumo
\begin{resumo}
 
O desenvolvimento ágil está cada dia mais presente no cotidiano das empresas desenvolvedoras de software. A crescente busca por agilidade no desenvolvimento e a competitividade do mercado impactam na existência deste cenário. Portanto, muitas empresas buscam aplicar as metodologias definidas neste ramo de desenvolvimento em seu processos, porém, essa não é uma tarefa simples. A definição e implantação dessas práticas realizadas de maneira não suficientemente adequada podem trazer resultados adversos ao esperado. Portanto este trabalho teve como objetivo implantar a utilização de uma ferramenta de integração contínua em uma fábrica de software, o Núcleo de Práticas em Informática (NPI) da UFC - Campus Quixadá. Uma ferramenta de integração contínua está inserida em uma das práticas definidas pelo Extreme Programming (XP). Para a realização deste objetivo fora analisado o processo vigente executado no NPI, além da experiência do autor que era um membro desta fábrica de software. Somado a esta etapa foi definida a ferramenta de integração contínua a ser implantada de acordo com um conjunto de característica especificadas, o conhecimento dos membros da fábrica acerca de Gerência de configuração e Integração Contínua foi avaliado, e por fim, a implantação, com a definição de pontos positivos e negativos encontrados na implantação.


 \textbf{Palavras-chaves}: Integração Contínua. Desenvolvimento Ágil. Gerenciamento de Configuração.
\end{resumo}
%% resumo em inglês
\begin{resumo}[Abstract]
 \begin{otherlanguage*}{english}
   This is the english abstract.

   \vspace{\onelineskip}
 
   \noindent 
   \textbf{Key-words}: Continuous Integration. Software Quality. Configuration Management.
 \end{otherlanguage*}
\end{resumo}
%\include{resumo/fr}
%\include{resumo/es}

% Lista de ilustrações
\pdfbookmark[0]{\listfigurename}{lof}
\listoffigures*
\cleardoublepage

% Lista de tabelas
\pdfbookmark[0]{\listtablename}{lot}
\listoftables*
\cleardoublepage

% Abreviaturas e Siglas
%% Lista de abreviaturas e siglas
% ---
\begin{siglas}
  \item[NPI] Núcleo de Práticas em Informática
  \item[CMMI] Capability Maturity Model Integration
  \item[MPS.BR] Melhoria de Processo Brasileiro
  \item[GC] Gerência de Configuração
  \item[SCV] Sistema de Controle de Versão
  \item[SCM] Sistema de Controle de Mudança
  \item[IC]  Integração Contínua
  \item[UFC] Universidade Federal do Ceará
  \item[TI] Tecnologia da Informação
  \item[SGBD] Sistemas de Gerência de Banco de Dados
  \item[CCB] Change Control Board
  \item[SBC] Sociedade Brasileira de Computação
\end{siglas}
% ---

% Símbolos
%\include{editaveis/simbolos}

% Sumário
\pdfbookmark[0]{\contentsname}{toc}
\tableofcontents*
\cleardoublepage

% ----------------------------------------------------------
% ELEMENTOS TEXTUAIS
% ----------------------------------------------------------
\textual

% ----------------------------------------------------------
% Introdução (exemplo de capítulo sem numeração, mas presente no Sumário)
% ----------------------------------------------------------
\chapter{Introdução}
% ----------------------------------------------------------

O presente trabalho visa aplicar no Núcleo de Práticas em Informática da Universidade Federal do Ceará do Campus de Quixadá (NPI) a utilização de uma ferramenta de Integração Contínua (IC), buscando melhorar as taxas de manutenibilidade dos sistemas de software produzidos e avaliar o impacto que esta ferramenta possui sobre a manutenibilidade de um sistema de software.

O NPI é o local onde estudantes que estão em ano de conclusão de curso podem estagiar e aprimorar seus conhecimentos adquiridos no decorrer do curso além de concluir seus componentes curriculares obrigatórios. O NPI surgiu devido à pouca demanda de empresas de Tecnologia da Informação (TI) na região onde a universidade se encontra, em Quixadá no Ceará. Dentro do NPI, os projetos desenvolvidos têm como objetivo construir soluções que facilitem as atividades do cotidiano da universidade, esta que tem um grande interesse no desenvolvimento destes projetos, pois consegue reduzir custos ao priorizar construções de sistemas internamente. Em paralelo, aumenta a qualidade dos profissionais formados por ela, além de proporcionar um ambiente real de trabalho que facilita a entrada dos concludentes no mercado de trabalho \cite{npi2013}.

Dentro do NPI existe um modelo de processo definido em que os desenvolvedores devem seguir para o exercício de suas atividades \cite{npi2013}. Entretanto, este não é devidamente seguido, ocasionando uma despadronização na maneira como estes  desenvolvedores trabalham em seus projetos. Somado-se a isto, o NPI apresenta problemas tais como: "Baixa qualidade da documentação dos sistemas; [\ldots] Falta de uma equipe de manutenção; [\ldots] Rotatividade dos profissionais" o que acaba gerando graves problemas de manutenção \cite[p.~4]{paduelli2006}.


Segundo \citeonline[p.~170]{sommerville2011}, "a manutenção de software é o processo geral de mudança em um sistema depois que ele é liberado para uso". Sendo assim, entende-se como manutenção de software  qualquer alteração realizada no sistema após este ser considerado "pronto" e implantado em seu ambiente de operação. Atualmente, esta vem ganhando uma maior atenção por parte das empresas desenvolvedoras de software. Isso acontece devido aos altos custos na fase de manutenção, podendo atingir 70\% do esforço total aplicado no projeto, além de sofrer possíveis aumentos ao longo da produção do software \cite{pressman2010}.

Mudanças no software são inevitáveis, e não possuem regra, simplesmente acontecem. Garantir que essas mudanças sejam devidamente controladas, identificadas é o papel da  Gerência de Configuração (GC). A importância da GC fica evidenciada quando diferentes modelos de maturidade o abordam como o MPS.BR no nível F e o CMMI(Capability Maturity Model Integration) \cite{furlaneto2006}.

A experiência em aplicar ferramentas  de gestão de configuração com o objetivo de melhorar a manutenibilidade de uma fábrica de software é abordada através de um conjunto de ferramentas automatizadas que buscam melhorar a manutenibilidade do software \cite{poliana2005}. Diferentemente do trabalho a ser desenvolvido nesse projeto, que busca verificar o impacto de uma ferramenta em específico, de Integração Contínua,  na melhoria da manutenibilidade.

Entende-se Integração Contínua como uma ferramenta de gestão de configuração, que auxilia os desenvolvedores e permite que as mudanças realizadas no software sejam imediatamente avaliadas, testadas, verificadas de modo a prover um  \textit{feedback} imediato para correção de possíveis erros de integração que somente seriam verificados futuramente após problemas mais complexos de integração \cite{paul2007}.

\citeonline{furlaneto2006} relata a construção de uma ferramenta própria e um modelo de processo, que apoiasse a gestão de configuração. Ferramentas diversas foram avaliadas e comparadas com a ferramenta desenvolvida, verificando-se, que a ferramenta desenvolvida atendeu bem o escopo definido pelo MPS.BR (Melhoria de Processo Brasileiro) e que após ser comparada com as ferramentas existentes, esta conseguiu atender bem os requisitos definidos. O trabalho citado difere do proposto ao criar uma ferramenta que apoie a Gerência de Configuração, enquanto o proposto visa avaliar o impacto de uma ferramenta específica em projetos e pessoas.

A justificativa pela escolha do tema se dá pela ausência de pesquisas que tenham como objetivo avaliar o impacto que uma ferramenta de integração contínua exerce sobre a manutenibilidade de um produto de software. Gerar conhecimento para o mercado de modo que ajude empresas a avaliar a necessidade e a viabilidade do uso de ferramentas deste tipo.

% ----------------------------------------------------------
% PARTE I
% ----------------------------------------------------------
%\part{Preparação da pesquisa}

% Capitulo com exemplos de comandos inseridos de arquivo externo 
\chapter{Fundamentação Teórica}\label{fundamentacao}

\section{Manutenção de Software}\label{manutencao}
A manutenção de software é qualquer alteração em um sistema de software após a implantação em seu ambiente de operação. Todo software passa por mudanças para adaptar-se a outro sistema operacional, mudanças de requisitos, ou simplesmente correção de funcionalidades. Atualmente as empresas estão tendo uma maior atenção ao processo de manutenção de software segundo \apud{pigoski1997,polo2002}{figueiredo2005}. Este custo não se restringe a apenas termos financeiros, bem como retrabalho. O esforço de retrabalho ocorre pelo fato da maioria das equipes de manutenção não estarem relacionadas com a equipe de desenvolvimento e pelo fato da pouca atenção com a documentação do software, com o decorrer do tempo evoluções no software são realizadas e a documentação não é devidamente atualizada \cite{sergio2005}.

\subsection{Tipos de Manutenção}
Como citado anteriormente todo software passa por mudanças, e essas mudanças sendo realizadas após a entrega do software caracterizam a atividade de manutenção. "Um software não se desgasta como peças de um equipamento, mas se deteriora no sentido de os objetivos de suas funcionalidades cada vez menos se adequarem ao ambiente externo" \space  \cite[p.~33]{matheus2007}.

Os tipos de manutenção definidas são: \apud{lientz1980}{matheus2007} \textit{corretivas}, \textit{adaptativa} e \textit{perfectivas} 
\subsubsection{Manutenção Corretiva}
Manutenções corretivas visam corrigir defeitos funcionais, onde uma determinada funcionalidade do sistema se comporta de maneira diferentes da especificada para aquela funcionalidade.
\apudonline{pfleeger2001}{matheus2007} relata um problema de impressão de um relatório, em que as linhas que eram impressas por cada folha eram maiores do que foi especificado, sobrepondo as informações das outras linhas. O problema foi identificado como uma falha de driver da impressora, e foi preciso alterar o menu de impressão para adicionar um novo parâmetro, este que iria referenciar o número de linhas que seria impresso.
\subsubsection{Manutenção Adaptativa}
As manutenções do tipo adaptativa retratam a alteração do software de modo a adaptá-lo a um novo ambiente de execução. Por exemplo, alterar o sistema devido a uma nova lei, que força o sistema a se \textbf{adaptar} ao ambiente externo \apudonline{pfleeger2001}{matheus2007} aborda uma situação onde havia um Sistema de  Gerência de Banco de Dados (SGBD) que foi atualizado, e as rotinas de acesso ao disco também foram alteradas, necessitando de uma parâmetro adicional. Este tipo de manutenção tem como característica não a correção de defeitos, até por que não existia, mas adaptá-lo ao novo ambiente de operação.
\subsubsection{Manutenção Perfectiva}
Manutenções perfectivas tem como objetivo adicionar funcionalidades ao sistemas, seja para obter um \textit{business value} maior ao produto de software, para competir com um software concorrente no mercado, ou simplesmente para atender uma solicitação de usuário. Este processo de mundaça é realizado mediante uma avaliação prévia  do sistema, que visa avaliar se a arquitetura do sistema suporta as novas funcionalidades sem degradá-la.
\subsection{Custos de Manutenção}
Para toda empresa, quanto menos custos melhor, mas o cenário atual nos diz que os maiores custos em produto de software estão na fase de manutenção\apud{pigoski1997,polo2002}{figueiredo2005}
e não alterar o software de maneira rápida o suficiente pode gerar grandes prejuízos. Os altos custo relacionados a manutenção estão inerentes a manutenção em si, esta atividade lida com diversos problemas com má documentação do software a ser mantido, e a imprevisibilidade, não saber em que estado o sistema foi construído, sobre que técnicas, padrões, metodologias, etc. 
\subsection{Documentação de Manutenção}

\subsection{Plano de Gerenciamento de Configuração}
O Plano de Gerenciamento de Configuração (PGC) descreve todas as atividades de configuração e mudança que serão realizadas durante o projeto. Um conjunto de atividades, responsabilidades, ferramentas, recursos e etc. A gerência de configuração tem como objetivo garantir a integridade dos itens de configuração, que são qualquer artefato que esteja sob custódia da Gerência de Configuração, através do versionamento, da identificação, controlando mudanças e acesso. 
\section{Gerência de Configuração}
A gerência de configuração é a área da engenharia de software responsável pela evolução do software. Ela atua durante todo o ciclo de vida do produto de software e, por meio de técnicas, ferramentas e metodologias, visa garantir que as mudanças que irão ocorrer dentro do ciclo de vida do desenvolvimento do software sejam identificadas, avaliadas e comunicada a todos os envolvidos através de ferramentas que auxiliam neste processo de evolução.
Portanto "o propósito do processo de Gerência de Configuração é estabelecer e manter a integridade de todos os produtos de trabalho de um processo ou projeto e disponibilizá-la a todos os envolvidos"\space\cite{mpsbr}.
\subsection{Sistema de Controle de Versão}
Um sistema de controle de versão: "	[...] combina procedimentos e ferramentas para gerenciar diferentes versões de objetos de configurações que são criadas durante o processo de engenharia de software" \cite[p.~927]{pressman2010}.
Atualmente, o uso de sistema de controle de versão se tornou comum nas empresas de grande e pequeno porte. Tais ferramentas permitem que se tenha o controle de diferentes versões de arquivos que estão submetidos ao versionamento, recuperação de versões antigas, visualizar alterações realizadas em arquivos e saber por quem e quando o arquivo foi alterado. Através de comandos (i.e.,\textit{check-in},\textit{check-out}) os usuários conseguem se comunicar com o repositório a fim de obter os artefatos ali armazenados \cite{gleiph2011}. Em situações especais faz-se necessário que os desenvolvedores trabalhem em una linha diferente da original chamada de \textit{mainline}, geralmente essa situação ocorre quando tem-se como objetivo consertar bugs de versões anteriores do repositório, nesse caso um \textit{branch}, uma ramificação na linha de desenvolvimento do controle de versão que permite o trabalho em paralelo sobre o mesmo repositório.
\begin{figure}[tbh]
\centering
\caption[Branch no Sistema de Controle de Versão]{Branch no Sistema de Controle de Versão}
\includegraphics[width=0.7\linewidth]{./images/branch}
\label{fig:Branch}
\legend{Fonte: \citeonline{tableless2012}}
\end{figure}
A figura \autoref{fig:Branch} demonstra a criação de um \textit{branch} paralelo a linha de desenvolvimento principal chamada de branch feature1 e branch master respectivamente, posteriormente ocorre a integração das ações realizar no \textit{branch feature1} é incorporado ao \textit{branch master}.
\subsubsection{Sistema de Controle de Versão Local}
Um sistema de controle de versão local	armazenam todas as informações de um arquivo submetido ao versionamento na máquina local, guardando diferentes versões    ,mx dcdx daquele arquivo localmente como demonstrado na figura \autoref{fig:SCVLocal}.
\begin{figure}[tbh]
\centering
\caption[Sistema de Controle de Versão Local]{Sistema de Controle de Versão Local}
\includegraphics[width=0.5\linewidth]{./images/scvlocal}
\label{fig:SCVLocal}
\legend{Fonte:\cite{git}}
\end{figure}
\subsubsection{Sistema de Controle de Versão Centralizado} Sistema de controle de versão centralizado como o nome diz possuem um único servidor centralizado, como o \textit{subversion} \footnote{http://subversion.apache.org}, \textit{perforce} \footnote{http://www.perforce.com}, este tipo de padrão de SCV mantém em seu único servidor todos os arquivos versionados. Para cada comando de comunicação realizado nos arquivos versionados, uma requisição deverá ser feita, podendo gerar lentidão ou deixar o servidor fora de funcionamento.
\begin{figure}[tbh]
\centering
\caption[Sistema de Controle de Versão Centralizado]{Sistema de Controle de Versão Centralizado}
\includegraphics[width=0.5\linewidth]{./images/scvcentral}
\label{fig:SCVCentral}
\legend{Fonte: \citeonline{git}}
\end{figure}
No exemplo acima dois desenvolvedores trabalhando em máquinas diferentes realizam a comunicação com o servidor central para obter o artefato de trabalho.	
\subsubsection{Sistema de Controle de Versão Distribuído}Os sistemas de controle de versão distribuído possuem um servidor central onde os arquivos são submetidos a versionamento, entretanto cada desenvolvedor possui em sua máquina de trabalho as versões que estavam no servidor, tornando cada \textit{workstation} um "servidor", portanto, caso ocorra um problema no servidor central, estes podem ser recuperados via \textit{workstation}, mantendo a integridade dos arquivos e evitando ser um ponto único de falha, como mostra a figura \autoref{fig:SCVDistribuido}:
\begin{figure}[tbh]
\centering
\caption[Sistema de Controle de Versão Distribuído]{Sistema de Controle de Versão Distribuído}
\includegraphics[width=0.5\linewidth]{./images/scvdist}
\label{fig:SCVDistribuido}
\legend{Fonte: \citeonline{git}}
\end{figure}
\subsection{Sistema de Controle de Mudança}
Todo software sofre mudanças, lidar com as mudanças é o papel da gerência de configuração, e para isso o gerente de configuração utiliza de um sistemas de controle de mudança. "O controle de mudança combina procedimentos humanos e ferramentas automatizadas para proporcionar um mecanismo de controle de mudança" \citeonline[p~.930]{pressman2010}. As mudanças devem ser avaliadas com cautela baseando-se, em seu custo benefício, uma combinação de esforço e \textit{business value}. A mudança tem início quando um "cliente" solicita a mudanças através de um formulário, conhecida com \textit{change request}. Nesse formulário é descrito os aspectos da mudança, após a solicitação ser realizada, esta deve ser avaliada, verificando se a mesma já foi solicitada, ou corrigida em caso de \textit{bugs}. Após a mudança ser validada, uma equipe de desenvolvedores e avaliam os impactos que esta mudança têm sobre o sistema, verificando custo/benefício e esforço de realização \cite{sommerville2011}. Posterior a essa análise, a mudança será avaliada por um comitê de controle de mudança (CCB) que avaliará o impacto da perspectiva do negócio, o que decidirá se esta mudança será revisada, aprovada ou reprovada. Alguns sistemas que fornecem este controle sobre as mudança são: \textit{redmine \footnote{http://www.redmine.org}, GitHub \footnote{http://www.github.com} Jira \footnote{https://www.atlassian.com/software/jira}}
\subsection{Auditoria de Configuração}
"Uma auditoria de configuração de software complementa a revisão técnica formal ao avaliar um objeto de configuração quanto às características que geralmente não são consideradas durante a revisão"\space\citeonline[p~.934]{pressman2010}. Ela tem como objetivo garantir que mesmo com as mudanças realizadas a qualidade foi mantida. As auditorias se dividem em dois tipos: auditorias funcionais e auditorias físicas, a auditoria funcional baseia-se em verificar se os itens de configuração estão devidamente atualizados e se as práticas e padrões foram realizados da maneira correta, enquanto a auditoria funcional, busca verificar os aspectos lógicos dos itens de configuração.
\subsection{Ferramentas de Build}
As ferramentas de build tem como objetivo automatizar processos repetitivos, aumentando a produtividade e facilitando o trabalho do desenvolvedor. Através da definição de uma rotina, ou conjunto de comandos, o desenvolvedor informa a ferramenta que tipo de processo ele deseja automatizar, pode ser desde compilar e testar uma classe, como dropar e criar uma tabela nova no banco de dados, comprimir arquivos css e javascript, cabe ao desenvolvedor definir o escopo da automatização. Alguns exemplo deste tipo de ferramenta são: \textit{Ant, Grunt, Gulp, Maven}.


\begin{figure}[h]
\centering
\caption[Processo Lógico de uma Build]{Processo Lógico de uma Build}
\includegraphics[width=0.7\linewidth]{./images/build}
\label{fig:build}
\legend{Fonte: \citeonline{paul2007}}
\end{figure}
Na figura \autoref{fig:build} um script foi definido para realizar as seguintes funções, será realizado um clean no projeto, compilará o código fonte, integrará com o banco de dados, executará testes e inspeções no código e por fim irá dar o \textit{deploy} da aplicação.


%\subsection{Ferramentas de Integração Contínua}

\section{Integração Contínua}\label{integracaocont}
\begin{OnehalfSpace}
A integração contínua tem como objetivo identificar erros o mais rápido possível, ela permite que alterações efetuadas e integradas aos repositórios dos sistemas de controle de versão (SCV) sejam posteriormente verificadas e caso erros ocorram, este serão notificados imediatamente ao autor da alteração.
Entende-se Integração Contínua como:
\end{OnehalfSpace}

\begin{citacao}
"[...] uma prática de desenvolvimento de software onde os membros de um time integram seu trabalho frequentemente, geralmente cada pessoa integra pelo menos diariamente – podendo haver múltiplas integrações por dia. Cada integração é verificada por uma build automatizada (incluindo testes) para detectar erros de integração o mais rápido possível. Muitos times acham que essa abordagem leva a uma significante redução nos problemas de integração e permite que um time desenvolva software coeso mais rapidamente." \citeonline[p.~s/n, tradução nossa]{fowler2000}
\end{citacao}

\begin{figure}[tbh]
\centering
\caption[Ambiente de Integração Contínua]{Ambiente de Integração Contínua}
\includegraphics[width=0.7\linewidth]{./images/CI}
\label{fig:CI}
\legend{Fonte: \citeonline{paul2005}}
\end{figure}

A figura \autoref{fig:CI} descreve um ambiente em que um servidor de integração contínua é utilizado. Existem três ambientes de trabalho distintos formado por três desenvolvedores que obtiveram uma cópia do projeto do repositório do SCV para trabalharem em suas \textit{workstation}, durante o trabalho alterações foram realizadas e commitadas ao repositório central, após a inserção junto ao repositório o servidor de integração contínua verifica as alterações e executa uma build de integração,caso exista um problema com a build e esta quebre, o responsável pela alteração será informado sobre a quebra e terá como objetivo consertar a build.

As principais vantagens em utilizar um servidor de integração contínua segundo \citeonline[p.~29]{paul2007} são:

\begin{itemize}
\item Redução de Riscos.
\item Redução de processos manuais repetitivos.
\item Permitir melhor visibilidade do projeto.
\item Estabelecer uma maior confiança no produto do time de desenvolvimento.
\end{itemize}

%\subsection{Integração Contínua e a Redução de Riscos}
%Os Riscos em produtos de software estão diretamente relacionados. Segundo \citeonline[p~.48]{paul2007} se você consegue reduzir certos riscos no software, você pode melhorar a qualidade do software.
\subsection{Builds Automatizadas}
Builds são rotinas de execução definidas com o objetivo de reduzir processos repetitivos. Durante o processo de desenvolvimento de um software muitas ações tendem a serem repetidas por parte dos desenvolvedores, utilizar o tempo para a realização  de atividades que poderiam ser automatizadas, de forma manual, reduz a produtividade e preocupações com melhorias devido ao tempo "apertado". Somando-se a isso, uma build garante que tudo que está nela definido será executado, evitando assim, que determinada ação seja esquecida, ou caso um novo membro entre na equipe uma explicação do que ele deve fazer, ou não esquecer de fazer, não faz-se necessário.

\subsection{Integração Contínua Manual}
Na IC manual o processo de integração é realizado individualmente, possibilitando que 
apenas um desenvolvedor realize check-in no repositório durante o intervalo de integração \citeonline{gleiph2011}. Este tipo de abordagem como permite que apenas uma pessoa realize o \textit{check-in}, as integrações serão contínuas e seguidas e não paralelas, este tipo de abordagem garante uma maior confiabilidade das integrações, pois segue um padrão de integração, os itens do repositório possuem maior consistência e a garantia da estrutura do repositório é mantida. \cite{gleiph2011}

\subsection{Integração Contínua Automatizada}
A integração contínua automatizada é auxiliada pelo uso de um servidor de integração contínua, que obtém do controle de versão as alterações realizadas e executada sua build privada afim de verificar possíveis erros gerados por essas modificações. Ver \autoref{integracaocont} \autoref{fig:CI} 
\begin{citacao}
 "IC Automática possui a vantagem de ser escalável 
e,  deste  modo,  oferecer  maior  suporte  ao  trabalho  colaborativo.  Com  a  utilização  de 
Servidores  de  IC,  a  responsabilidade  de  realizar  construções  da  integração  é  retirada  dos desenvolvedores. Portanto, os desenvolvedores podem realizar  check-in  sem a necessidade de 
conquistar a vez de integrar. Esse fator é fundamental para que os  check-ins  continuem sendo 
verificados  sem  a  necessidade  de  um desenvolvedor  realizar  a  construção  e identificar 
problemas, resultando na eliminação do gargalo humano \citeonline{gleiph2011}." 
\end{citacao}


%\subsection{Tipos de Build}
%\subsubsection{Builds Privadas}
%\subsubsection{Build de Integração}
%\subsubsection{Builds de Release}
%\subsubsection{Mecanismos de Build}
%\subsubsection{Builds Engatilhadas}
\chapter{Trabalhos Relacionados}\label{trabalhorel}
Nesta seção será descrito trabalhos que influenciaram os conceitos envolvidos neste trabalho além de demonstrar pontos comuns e distintos entre si e o proposto.
%\include{abntex2-modelo-include-comandos}
%\chapter{Objetivos}\label{objetivos}
\section{Objetivo Geral}\label{objetivoger}
Implantar e avaliar os impactos que o uso de uma ferramenta de integração contínua exerce sobre a manutenibilidade de um software.

\section{Objetivos Específicos}

\begin{itemize}
\item Avaliar o nível manutenibilidade do software produzido.
\item Implantar um processo de utilização de ferramenta de integração contínua.
\item Avaliar resultados obtidos.
\end{itemize}
% ----------------------------------------------------------
% PARTE II
% ----------------------------------------------------------
%\part{Referenciais Teóricos}


% Capitulo de revisão de literatura
%\chapter{Lorem ipsum dolor sit amet}
\chapter{Procedimentos Metodológicos}\label{metodologia}

\section{Analisar as atividades do Núcleo de Práticas}
Esta atividade visa identificar como as atividades ocorriam dentro do Núcleo de Práticas em Informática, para tal foi analisado o processo\footnote{http://www.npi.quixada.ufc.br/processo/} existente modelado pela ferramenta \textit{EPF Composer}. Além da experiência do autor pois este era um funcionário da organização com experiência de 8 meses nas atividades lá realizadas. 

\section{Pesquisar e selecionar a ferramenta de Integração Contínua}

Esta atividade consiste em  colher informações e selecionar a ferramenta que melhor se adapta a realidade existente no Núcleo de Práticas em Informática. 
Para a escolha da ferramenta foi preciso definir um conjunto de requisitos que a ferramenta deveria possuir e suprir, de modo a filtrar a ferramenta escolhida dentre as diversas existentes. As definições foram descritas \autoref{escolhaFerramenta}. 

\section{Absorção do perfil dos estagiários do Núcleo de Práticas}
Esta atividade tem como objetivo entender e identificar os conhecimentos dos estagiários do núcleo acerca de integração contínua, experiência de uso em projetos pessoais, conhecimento na ferramenta e como esta funciona.

\section{Implantação da ferramenta de integração contínua}
A implantação será realizada no núcleo e a utilização da ferramenta em um projeto piloto.


\pagebreak
\section{Cronograma de Execução}

\begin{table}[!htpb]
\centering
\caption{Cronograma das atividades previstas}

% definindo o tamanho da fonte para small
\begin{small} 
  
% redefinindo o espaçamento das colunas
\setlength{\tabcolsep}{6pt} 

% \cline é semelhante ao \hline, porém é possível indicar as colunas que terão essa a linha horizontal
% \multicolumn{10}{c|}{Meses} indica que dez colunas serão mescladas e a palavra Meses estará centralizada dentro delas.

\begin{tabular}{|c|c|c|c|c|c|c|c|c}\hline
 & \multicolumn{7}{c|}{2014}\\ \cline{2-8}
\raisebox{1.5ex}{ATIVIDADES} & Mai & Jun & Jul & Ago & Set & Out & Nov \\ \hline

Estudo de Campo & X & X & & &  &  & \\ \hline
Defesa do Projeto &  & X & & &  &  & \\ \hline
Avaliação do Processo do NPI &  &  & X & X &  &  & \\ \hline
Avaliação da Manutenibilidade do GAL &  &  &  & X & X &  & \\ \hline
Implantação da Integração Contínua &  &  &  & X & X &  & \\ \hline
Treinamento do uso da Ferramenta &  &  &  & & X & X & \\ \hline
Avaliação da Manutenibilidade do GAL  &  &  &  &  &  & X & \\ \hline
Análise do Dados de Manutenibilidade &  &  &  & &  & X &  \\ \hline
Revisão final da monografia & & & & & & X & X \\ \hline
Defesa do Projeto & & & & & & & X \\ \hline

\end{tabular} 
\end{small}
\legend {Fonte: Elaborado pelo Autor}
\label{t_cronograma}
\end{table} 


% ----------------------------------------------------------
% PARTE III
% ----------------------------------------------------------
%\part{Resultados}
\chapter{Desenvolvimento/Resultados}
Esta seção tem como objetivo apresentar as etapas para a elaboração deste trabalho. A seção é composta de cinco subseções. A \autoref{Analise-NPI} retrata a primeira fase do trabalho onde é explanado as atividades do NPI. A \autoref{pesq-selecao} descreve como a ferramenta de integração contínua foi escolhida e sob quais critérios.


\section{Análise das atividades do Núcleo de Práticas em Informática}\label{Analise-NPI}
A análise de como as atividades eram executadas dentro do NPI foi primeiramente analisada de acordo com o processo definido, disponível no site do NPI \footnote{www.npi.quixada.ufc.br/processo/}, o qual regula como as atividades ocorrem. As atividades e o processo baseia-se no SCRUM e nas metodologias ágeis como equipes de pequeno número de componentes \textit{Sprint Planning}, \textit{Product Backlog}, \textit{Sprint Review}.

O NPI subdivide-se em dois turnos, manhã e tarde, sendo cada turno supervisionado por um professor orientador diferente. Estes turnos podem ou não estar trabalhando no mesmo projeto, embora o mais comum é que trabalhem em projetos diferentes. As equipes contam com em média oito membros onde comumente destes, dois são alocados para as atividades de requisitos e testes, um para liderança técnica, enquanto o restante da equipe é alocado para as atividades de desenvolvimento, incluindo o líder técnico. O professor supervisor tem como papel o auxílio aos líderes técnicos, acompanhamento do projeto, avaliação dos estagiários, escolha dos projetos a serem desenvolvidos pelas equipes e usualmente realizar o papel de \textit{Product Owner}. 

O líder técnico possui papel gerencial bem como de desenvolvimento, sua atribuições partem desde a condução de reuniões, resolução de conflitos,atribuição e definições de tarefas, até o acompanhamento das atividades. 

A \autoref{fig:processo-npi} mostra o processo utilizado no NPI modelado através da ferramenta EPF Composer. Na figura	existem duas atividades que ocorrem em paralelo, são elas: Avaliação do Processo e Iniciar Projeto, este que subdividi-se em mais três atividades, a primeira delas a atividade de Requisitos, que posteriormente fornece entrada para um ciclo de \textit{Sprints} que ocorrerá enquanto houver funcionalidades não implementadas, simultaneamente com a atividade de Requisitos estão de Gerenciamento do Projeto e o Gerenciamento de Configuração.
\begin{figure}[H]
\centering
\caption[Processo do NPI]{Processo do NPI.}
\includegraphics[scale=0.8]{./images/processo-npi}
\label{fig:processo-npi}
\legend {\fontsize{10}{12}\selectfont {Fonte: \citeonline{processonpi}}.}
\end{figure}

\section{Pesquisa e Seleção da Ferramenta}\label{pesq-selecao}
O processo de escolha da ferramenta de integração contínua teve como primeiro critério estar em acordo com a realidade das atividades executadas no NPI. Consequentemente o primeiro ponto a ser considerado foi o suporte que a ferramenta deveria prover as linguagens utilizadas no NPI, a linguagem Java. Outro aspecto considerado está relacionado com o custo de aquisição, esta não poderia ser paga ou deveria possuir uma versão \textit{free} que atendesse a demanda das atividades. Extensibilidade, devido as constantes mudanças de tecnologias utilizadas, fornecer suporte a diferentes linguagens, ferramentas, faz-se essencial; usabilidade, pois o NPI não conta com um Gerente de Configuração, sendo assim esta tarefa de manter uma integração contínua deve ser facilitada ao máximo por meio de sua usabilidade, tais como, \textit{inteligibilidade}, \textit{apreensibilidade}; possuir segurança adequada, definição de usuário, papéis.

Deste modo a ferramenta escolhida foi o Jenkins , anteriormente conhecido como Hudson, é uma ferramenta de integração contínua \textit{open source}, que fornece suporte a projetos de diferentes linguagens e tecnologias ,.NET, Ruby, Grails, PHP, bem como Java, linguagem base de sua construção. \cite{smart2011}, e como esta preenche os requisitos definidos será descrito abaixo.

\begin{itemize}
\item {\textbf{Suporte a Linguagem:}}

O Jenkins permite suporte a uma grande gama de linguagens, tais como Java, PHP, Rails, Grails, Python, entre outras.

\item {\textbf{Suporte ao Sistema de Controle de Versão:}}
O Jenkins consegue integrar nativamente com os principais sistemas de controle de versão tais como: \textit{CVS}, \textit{SVN},  \textit{Mercurial}, e o \textit{Git} através da utilização de plugin.


\item {\textbf{Segurança:}}
A segurança do Jenkins é habilitada através de permissões e papéis, onde a base de dados de usuários pode ser pela base interna do Jenkins, LDAP, usuários do sistema operacional e também através do usuário vinculado ao GitHub.

\item {\textbf{Extensibilidade:}}
Jenkins é extremamente flexível e adaptável, permitindo assim oferecer uma melhor atuação para diferentes propósitos, através das centenas de plugins disponíveis. Plugins este que oferecem tudo desde sistemas de controle de versão, ferramentas de build, ferramentas de análise estática de código, notificadores de build, alterações de UI, integração com sistemas externos (\textit{Jira, Redmine}) \cite{smart2011}.


\item {\textbf{Usabilidade:}}
"Primeiramente, Jenkins é fácil de usar. A interface é simples e intuitiva, e o Jenkins como um todo possui uma curva de aprendizado baixa" \citeonline[p~.3]{smart2011}.

\item {\textbf{Instalação e Configuração:}}

Facilidade de instalação, diferentes ambiente de operação, tais como sistemas operacionais, utilização de recursos. Documentação clara e objetiva do processo de instalação informando dependência existentes.

\end{itemize}

\section{Perfil dos Estagiários e conhecimentos sobre Integração Contínua}
Entender os conhecimentos dos estagiários do NPI acerca do entendimento, funcionalidade e como esta mudaria suas rotinas de trabalho foi essencial para um entendimento e aperfeiçoamento do processo de implantação da ferramenta.

Para tal, fora realizado um questionário fechado, distribuído de maneira eletrônica para todos os estagiários do NPI. Embora todos não tenham respondido, uma boa amostra foi obtida em confronto com o número total de estagiários em atividade. O referido questionário será apresentado abaixo.

\pagebreak
\begin{table}[htb]
\centering
\caption[Conhecimentos em Integração Contínua]{Conhecimento em Integração Contínua.}
\label{tab-ic}
\begin{tabular}{p{5.0cm}l|p{5.0cm}|p{5.40cm}|p{5.40cm}}
  \hline
   \textbf{Perguntas} & \textbf{Opções de Respostas}\\
    \hline
    & Testador \\
    Qual a sua função no NPI? & Engenheiro de Requisitos \\
    & Testador \\
    & Líder Técnico / Gerente \\
    \hline
    Você sabe o que é Integração Contínua? & Sim \\
    & Não \\
    \hline
    Você já utilizou Integração Contínua em algum projeto? & Sim \\
    & Não \\
    \hline
    Você conhece ou utilizou alguma destas ferramentas de Integração Contínua?  & \\
    & Atlassian Bamboo \\
    & Apache Continuum \\
    & CruiseControl \\
    & Jenkins / Hudson \\
    & Outra \\
    & Desconheço ou nunca utilizei nenhuma delas \\
	\hline
	Você sabe o que é Gerência de Configuração? & Sim \\
	& Não \\
	\hline
\end{tabular}
\legend {\fontsize{10}{12}\selectfont {Fonte: Elaborado pelo autor}.}
\end{table}

Ao todo, vinte e três estagiários participaram da pesquisa, quase o NPI em sua totalidade, as devidas respostas serão exibidas abaixo na ordem em que as perguntas foram apresentadas aos questionados. A elaboração deste questionário teve como objetivo gerar dados quantitativos de modo a entender o perfil dos estagiários do NPI, facilitando assim o processo de implantação da ferramenta de integração contínua. Onde esses dados geraram conhecimentos para apresentação e explicação as equipes de forma mais proveitosa e focada nas dificuldades.


\begin{figure}[H]
\centering
\caption[Função NPI]{Função no NPI.}\includegraphics[scale=0.9]{./images/grafico-ci01}
\label{fig:grafico01-npi}
\legend {\fontsize{10}{12}\selectfont {Fonte: Elaborado pelo autor}.}
\end{figure}

A \autoref{fig:grafico01-npi} demonstra que a grande maioria dos estagiários do NPI estão alocados para atividades exclusivas de desenvolvimento, posteriormente atividade de engenharia de requisitos, líderes técnicos e gerentes. Este resultado obtido por meio das respostas ajudou a elucidar os conhecimentos e os tipos de conhecimentos predominante nos estagiários do núcleo.

\begin{figure}[H]
\centering
\caption[Conhecem Integração Contínua]{Conhecem Integração Contínua.}\includegraphics[scale=0.9]{./images/grafico-ci02}
\label{fig:grafico02-npi}
\legend {\fontsize{10}{12}\selectfont {Fonte: Elaborado pelo autor}.}
\end{figure}

Como descrito na \autoref{fig:grafico02-npi}, a maioria dos questionados conheciam o que era uma ferramenta de integração contínua, e como esta funcionava, embora esta não tenha sido uma superioridade notável, facilitou o processo de implantação em razão dos conhecimentos prévios dos estagiários a cerca do assunto, permitindo assim uma menor rejeição na implantação devido ao conhecimentos dos benefícios que este tipo de ferramenta causaria ao projeto.

\begin{figure}[H]
\centering
\caption[Utilização em Projetos Pessoais]{Utilização em Projetos Pessoais.}\includegraphics[scale=0.9]{./images/grafico-ci03}
\label{fig:grafico03-npi}
\legend {\fontsize{10}{12}\selectfont {Fonte: Elaborado pelo autor}.}
\end{figure}

Embora a maioria dos estagiários conheça a ferramenta, pouco mais de 4\% dos questionados utilizaram a integração contínua de forma prática, isto é, enfrentaram o impacto de sua utilização. Seja através do \textit{feedback} imediato fornecido pela ferramenta, ou pela alteração de seus processos de trabalho.	

\begin{figure}[H]
\centering
\caption[Conhecimento e Utilização de Ferramentas de Integração Contínua]{Conhecimento e Utilização de Ferramentas de Integração Contínua.}\includegraphics[scale=0.9]{./images/grafico-ci04}
\label{fig:grafico04-npi}
\legend {\fontsize{10}{12}\selectfont {Fonte: Elaborado pelo autor}.}
\end{figure}
O gráfico da \autoref{fig:grafico04-npi} contrasta com o gráfico anterior, onde a maioria desconhece ou nunca utilizou nenhuma ferramenta, e dentre a única ferramenta citada, o Jenkins / Hudson, enquanto um questionado citou outra ferramenta mas não especificou qual seria esta. De todo modo a familiarização de alguns questionados com a ferramenta facilitará o processo de aceitação desta por parte dos membros, e gerará uma unificação de conhecimento, pois todos os membros irão trabalhar e conhecer apenas uma ferramenta, no caso o Jenkins.

 

\begin{figure}[H]
\centering 
\caption[Conhecem Gerência de Configuração]{Conhecem Gerência de Configuração.}\includegraphics[scale=0.9]{./images/grafico-ci05}
\label{fig:grafico05-npi}
\legend {\fontsize{10}{12}\selectfont {Fonte: Elaborado pelo autor}.}
\end{figure}



\section{Processo de Implantação da Ferramenta de Integração Contínua}
Esta atividade tem como objetivo explicar como o processo ocorreu, apresentando o contexto do projeto, pontos negativos e positivos da implantação e aspectos a serem melhorados. A distribuição do conteúdo se dará da seguinte forma: a \autoref{contexto} explicará o contexto das atividades, tecnologias utilizadas, a \autoref{gpa} apresentará o projeto piloto onde a ferramenta foi implantada.

\subsection{Contexto}\label{contexto}

\subsubsection{Projeto Gestão de Projetos Acadêmicos }\label{gpa}

%\chapter{Discussão}
Esta seção tem como objetivo explanar os resultados obtidos no desenvolvimento deste trabalho.
\section{Resultados}

Com a utilização do SonarQube, a coleta contínua das métricas pode ser verificado e avaliada a sua importância dentro da atuação de uma ferramenta de integração contínua. Métricas foram coletas e armazenadas. Observamos que os dados fornecidos pela ferramenta serviu de insumo para um pequena alteração do processo, por meio de uma atividade de correção de \textit{issues} no \textit{Redmine}, observando que os resultados desta atividade, melhoraram os índices do software. Através da atividade de coleta das métricas e soluções das \textit{issues}, diminuiu-se os riscos atrelados ao produto, por meio da manutenção preventiva.

Outro aspecto que foi comprovadamente identificado foi a comunicação da equipe. Com a implantação da ferramenta, os membros da equipe puderam saber as falhas de integração do software de modo imediato e, assim, corrigir os problemas que causavam a má integração.

A ausência de testes automatizados inviabilizou a verificação de problemas de integração de maneira automatizada, embora a ferramenta de integração contínua fornecesse um grande suporte e a ajuda a este tipo de atividade este não pode ser mensurado , utilizado e avaliado.





\chapter{Considerações Finais}\label{consideracoes-finais}

Com o mercado cada vez mais competitivo para consumir software, construir software com uma maior qualidade é essencial para a sobrevivência no mercado de trabalho. Por isso, muitas empresas estão investindo na utilização de ferramentas de integração contínua no desenvolvimento de seus produtos. Entretanto, realizar uma implantação de uma ferramenta desta natureza, requer cuidados, pois caso esta não seja suficientemente bem planejada, poderá trazer resultados adversos ao esperado, pois uma implantação forçada ou mal sucedida pode gerar resistência da equipe, bem como diminuir a produtividade da equipe, burocratizar o processo, gerando assim um mal estar entre os membros do time e perca na qualidade do produto.

O trabalho proposto visou implantar uma ferramenta de integração contínua no Núcleo de Práticas em Informática da Universidade Federal do Ceará do Campus de Quixadá. O processo de implantação se iniciou a partir da extração de como os softwares eram desenvolvidos no NPI, obtido através da experiência do autor, e da análise do processo vigente. Somado-se a isto, fora pesquisada e selecionada uma ferramenta de integração contínua adaptada ao desenvolvimento praticado no NPI. Depois de escolhida a ferramenta, esta foi devidamente implantada e assim, conseguiu-se obter pontos positivos e negativos da implantação. Verificou-se que a utilização da coleta contínua das métricas foi essencial na melhoria da qualidade do produto.


O trabalho realizado por \citeonline{pereira2013} serviu como sabe para este trabalho, pois ajudou com conhecimentos sobre integração contínua este trabalho diverge nos pontos positivos e negativos identificados, pois era aplicado em um contexto de ferramentas semelhantes, mas em um organização com um nível de maturidade mais elevado, e converge na abordagem utilizada, e na condução do trabalho.
 

As limitações deste trabalho são a implantação de uma ferramenta de integração contínua em um ambiente de pequeno porte, onde seu impacto não pode ser validado em um ambiente maior de desenvolvimento. Além de uma validação científica mais criteriosa.

Como trabalhos futuros, sugerimos um acompanhamento de como a implantação desta ferramenta alterará o processo de desenvolvimento, e a coleta de dados que confirmem a melhoria dos sistema a longo prazo além da validação de uma integração contínua em um ambiente de projeto com testes automatizados.


Este trabalho permitiu a criação de uma cultura de integração contínua no NPI, possibilitando que os envolvidos visualizassem o impacto que esta ferramenta exerce no desenvolvimento do software e relatar as lições aprendidas que poderão ser utilizadas em futuras implantações.


% ---
% primeiro capitulo de Resultados
%\chapter{Lectus lobortis condimentum}

%\section{Vestibulum ante ipsum primis in faucibus orci luctus et ultrices posuere cubilia Curae}

%\lipsum[21-22]

% ---
% segundo capitulo de Resultados
%\chapter{Nam sed tellus sit amet lectus urna ullamcorper tristique interdum elementum}

%\section{Pellentesque sit amet pede ac sem eleifend consectetuer}

%\lipsum[24]

% ----------------------------------------------------------
% Finaliza a parte no bookmark do PDF para que se inicie o bookmark na raiz e adiciona espaço de parte no Sumário
% ----------------------------------------------------------
\phantompart

% ---
% Conclusão (outro exemplo de capítulo sem numeração e presente no sumário)
%\chapter*[Conclusão]{Conclusão}
%\addcontentsline{toc}{chapter}{Conclusão}

%\lipsum[31-33]

% ----------------------------------------------------------
% ELEMENTOS PÓS-TEXTUAIS
% ----------------------------------------------------------
\postextual

% Referências bibliográficas
\bibliography{bibtex/referencias}

% Glossário (Consulte o manual da classe abntex2 para orientações sobre o glossário)
%\glossary

% Apêndices
%\include{editaveis/apendices}

% Anexos
%\include{editaveis/anexos}

%---------------------------------------------------------------------
% INDICE REMISSIVO
%---------------------------------------------------------------------
\phantompart
\printindex
%---------------------------------------------------------------------
\end{document}
