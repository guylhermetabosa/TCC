\chapter{Considerações Finais}\label{consideracoes-finais}

Com o mercado cada vez mais competitivo para consumir software, construir software com uma maior qualidade é essencial para a sobrevivência no mercado de trabalho. Por isso, muitas empresas estão investindo na utilização de ferramentas de integração contínua no desenvolvimento de seus produtos. Entretanto, realizar uma implantação de uma ferramenta desta natureza, requer cuidados, pois caso esta não seja suficientemente bem planejada, poderá trazer resultados adversos ao esperado, pois uma implantação forçada ou mal sucedida pode gerar resistência da equipe, bem como diminuir a produtividade da equipe, burocratizar o processo, gerando assim um mal estar entre os membros do time e perca na qualidade do produto.

O trabalho proposto visou implantar uma ferramenta de integração contínua no Núcleo de Práticas em Informática da Universidade Federal do Ceará do Campus de Quixadá. O processo de implantação se iniciou a partir da extração de como os softwares eram desenvolvidos no NPI, obtido através da experiência do autor, e da análise do processo vigente. Somado-se a isto, fora pesquisada e selecionada uma ferramenta de integração contínua adaptada ao desenvolvimento praticado no NPI. Depois de escolhida a ferramenta, esta foi devidamente implantada e assim, conseguiu-se obter pontos positivos e negativos da implantação. Verificou-se que a utilização da coleta contínua das métricas foi essencial na melhoria da qualidade do produto.


O trabalho realizado por \citeonline{pereira2013} serviu como sabe para este trabalho, pois ajudou com conhecimentos sobre integração contínua este trabalho diverge nos pontos positivos e negativos identificados, pois era aplicado em um contexto de ferramentas semelhantes, mas em um organização com um nível de maturidade mais elevado, e converge na abordagem utilizada, e na condução do trabalho.
 

As limitações deste trabalho são a implantação de uma ferramenta de integração contínua em um ambiente de pequeno porte, onde seu impacto não pode ser validado em um ambiente maior de desenvolvimento. Além de uma validação científica mais criteriosa.

Como trabalhos futuros, sugerimos um acompanhamento de como a implantação desta ferramenta alterará o processo de desenvolvimento, e a coleta de dados que confirmem a melhoria dos sistema a longo prazo além da validação de uma integração contínua em um ambiente de projeto com testes automatizados.


Este trabalho permitiu a criação de uma cultura de integração contínua no NPI, possibilitando que os envolvidos visualizassem o impacto que esta ferramenta exerce no desenvolvimento do software e relatar as lições aprendidas que poderão ser utilizadas em futuras implantações.

