\chapter{Discussão}
Esta seção tem como objetivo explanar os resultados obtidos no desenvolvimento deste trabalho.
\section{Resultados}

Com a utilização do SonarQube, a coleta contínua das métricas pode ser verificado e avaliada a sua importância dentro da atuação de uma ferramenta de integração contínua. Métricas foram coletas e armazenadas. Observamos que os dados fornecidos pela ferramenta serviu de insumo para um pequena alteração do processo, por meio de uma atividade de correção de \textit{issues} no \textit{Redmine}, observando que os resultados desta atividade, melhoraram os índices do software. Através da atividade de coleta das métricas e soluções das \textit{issues}, diminuiu-se os riscos atrelados ao produto, por meio da manutenção preventiva.

Outro aspecto que foi comprovadamente identificado foi a comunicação da equipe. Com a implantação da ferramenta, os membros da equipe puderam saber as falhas de integração do software de modo imediato e, assim, corrigir os problemas que causavam a má integração.

A ausência de testes automatizados inviabilizou a verificação de problemas de integração de maneira automatizada, embora a ferramenta de integração contínua fornecesse um grande suporte e a ajuda a este tipo de atividade este não pode ser mensurado , utilizado e avaliado.



