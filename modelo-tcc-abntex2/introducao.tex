\chapter{Introdução}
% ----------------------------------------------------------

O presente trabalho visa aplicar no Núcleo de Práticas em Informática da Universidade Federal do Ceará do Campus de Quixadá (NPI) a utilização de uma ferramenta de Integração Contínua (IC), buscando melhorar as taxas de manutenibilidade dos sistemas de software produzidos e avaliar o impacto que esta ferramenta possui sobre a manutenibilidade de um sistema de software.

O NPI é o local onde estudantes que estão em ano de conclusão de curso podem estagiar e aprimorar seus conhecimentos adquiridos no decorrer do curso além de concluir seus componentes curriculares obrigatórios. O NPI surgiu devido à pouca demanda de empresas de Tecnologia da Informação (TI) na região onde a universidade se encontra, em Quixadá no Ceará. Dentro do NPI, os projetos desenvolvidos têm como objetivo construir soluções que facilitem as atividades do cotidiano da universidade, esta que tem um grande interesse no desenvolvimento destes projetos, pois consegue reduzir custos ao priorizar construções de sistemas internamente. Em paralelo, aumenta a qualidade dos profissionais formados por ela, além de proporcionar um ambiente real de trabalho que facilita a entrada dos concludentes no mercado de trabalho \cite{npi2013}.

Dentro do NPI existe um modelo de processo definido em que os desenvolvedores devem seguir para o exercício de suas atividades \cite{npi2013}. Entretanto, este não é devidamente seguido, ocasionando uma despadronização na maneira como estes  desenvolvedores trabalham em seus projetos. Somado-se a isto, o NPI apresenta problemas tais como: "Baixa qualidade da documentação dos sistemas; [\ldots] Falta de uma equipe de manutenção; [\ldots] Rotatividade dos profissionais" o que acaba gerando graves problemas de manutenção \cite[p.~4]{paduelli2006}.


Segundo \citeonline[p.~170]{sommerville2011}, "a manutenção de software é o processo geral de mudança em um sistema depois que ele é liberado para uso". Sendo assim, entende-se como manutenção de software  qualquer alteração realizada no sistema após este ser considerado "pronto" e implantado em seu ambiente de operação. Atualmente, esta vem ganhando uma maior atenção por parte das empresas desenvolvedoras de software. Isso acontece devido aos altos custos na fase de manutenção, podendo atingir 70\% do esforço total aplicado no projeto, além de sofrer possíveis aumentos ao longo da produção do software \cite{pressman2010}.

Mudanças no software são inevitáveis, e não possuem regra, simplesmente acontecem. Garantir que essas mudanças sejam devidamente controladas, identificadas é o papel da  Gerência de Configuração (GC). A importância da GC fica evidenciada quando diferentes modelos de maturidade o abordam como o MPS.BR no nível F e o CMMI(Capability Maturity Model Integration) \cite{furlaneto2006}.

A experiência em aplicar ferramentas  de gestão de configuração com o objetivo de melhorar a manutenibilidade de uma fábrica de software é abordada através de um conjunto de ferramentas automatizadas que buscam melhorar a manutenibilidade do software \cite{poliana2005}. Diferentemente do trabalho a ser desenvolvido nesse projeto, que busca verificar o impacto de uma ferramenta em específico, de Integração Contínua,  na melhoria da manutenibilidade.

Entende-se Integração Contínua como uma ferramenta de gestão de configuração, que auxilia os desenvolvedores e permite que as mudanças realizadas no software sejam imediatamente avaliadas, testadas, verificadas de modo a prover um  \textit{feedback} imediato para correção de possíveis erros de integração que somente seriam verificados futuramente após problemas mais complexos de integração \cite{paul2007}.

\citeonline{furlaneto2006} relata a construção de uma ferramenta própria e um modelo de processo, que apoiasse a gestão de configuração. Ferramentas diversas foram avaliadas e comparadas com a ferramenta desenvolvida, verificando-se, que a ferramenta desenvolvida atendeu bem o escopo definido pelo MPS.BR (Melhoria de Processo Brasileiro) e que após ser comparada com as ferramentas existentes, esta conseguiu atender bem os requisitos definidos. O trabalho citado difere do proposto ao criar uma ferramenta que apoie a Gerência de Configuração, enquanto o proposto visa avaliar o impacto de uma ferramenta específica em projetos e pessoas.

A justificativa pela escolha do tema se dá pela ausência de pesquisas que tenham como objetivo avaliar o impacto que uma ferramenta de integração contínua exerce sobre a manutenibilidade de um produto de software. Gerar conhecimento para o mercado de modo que ajude empresas a avaliar a necessidade e a viabilidade do uso de ferramentas deste tipo.