\chapter{Introdução}
% ----------------------------------------------------------

O presente trabalho visa implantar no Núcleo de Práticas em Informática da Universidade Federal do Ceará do Campus de Quixadá (NPI) a utilização de uma ferramenta de Integração Contínua (IC), e relatar a experiência obtida em sua implantação.

Entende-se Integração Contínua como uma ferramenta de gestão de configuração, que auxilia os desenvolvedores e permite que as mudanças realizadas no software sejam imediatamente avaliadas, testadas, verificadas de modo a prover um  \textit{feedback} imediato para correção de possíveis erros de integração que somente seriam verificados futuramente após problemas mais complexos de integração \cite{paul2007}.

Para esta experiência foi utilizado uma ferramenta de integração contínua automatizada, \textit{Jenkins}, como ambiente de aplicação o Núcleo de Práticas em Informática (NPI) da Universidade Federal do Ceará (UFC) do campus de Quixadá.

O NPI é o local onde estudantes que estão em ano de conclusão de curso podem estagiar e aprimorar seus conhecimentos adquiridos no decorrer do curso além de concluir seus componentes curriculares obrigatórios. Este surgiu devido à pouca demanda de empresas de Tecnologia da Informação (TI) na região onde a universidade se encontra, em Quixadá no Ceará.

Dentro do NPI, os projetos desenvolvidos têm como objetivo construir soluções que facilitem as atividades do cotidiano da universidade, esta que tem um grande interesse no desenvolvimento destes projetos, pois consegue reduzir custos ao priorizar construções de sistemas internamente \cite{npi2013}. Em paralelo, aumenta a qualidade dos profissionais formados por ela, além de proporcionar um ambiente real de trabalho que facilita a entrada dos concludentes no mercado de trabalho.

No NPI existe um modelo de processo definido em que os desenvolvedores devem seguir para o exercício de suas atividades \cite{npi2013}. Entretanto, este não é devidamente seguido, e não contempla dentro do processo a utilização de integração contínua, ocasionando uma despadronização na maneira como estes  desenvolvedores trabalham em seus projetos. Somado-se a isto, o NPI apresenta problemas tais como: "Baixa qualidade da documentação dos sistemas; [\ldots] Falta de uma equipe de manutenção; [\ldots] Rotatividade dos profissionais" \cite[p.~4]{paduelli2006}.

A experiência em aplicar ferramentas de gestão de configuração foi explanado por \citeonline{poliana2007}. Eles inseriram a utilização destas ferramentas para evitar a inserção de novos erros proveniente de manutenções realizadas no software. Diferentemente do trabalho a ser desenvolvido nesse projeto que busca relatar a experiência obtida na implantação de uma ferramenta de integração contínua.

Este trabalho tem como objetivo implantar uma ferramenta de integração contínua no Núcleo de Práticas em Informática da Universidade Federal do Ceará do campus Quixadá e como objetivos específicos estudar e analisar ferramentas de integração contínua e selecionar a que melhor se adapta ao Núcleo de Práticas em Informática;
selecionar e implantar uma ferramenta de integração contínua automatizada e por fim coletar relatos e resultados provenientes da implantação da ferramenta.
