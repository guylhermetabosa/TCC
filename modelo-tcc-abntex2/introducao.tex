\chapter{Introdução}
% ----------------------------------------------------------

O presente trabalho visa implantar no Núcleo de Práticas em Informática da Universidade Federal do Ceará do Campus de Quixadá (NPI) a utilização de uma ferramenta de Integração Contínua (IC), buscando analisar as taxas de manutenibilidade dos sistemas de software produzidos e avaliar o impacto que esta ferramenta possui sobre a manutenibilidade de um sistema de software.

O NPI é o local onde estudantes que estão em ano de conclusão de curso podem estagiar e aprimorar seus conhecimentos adquiridos no decorrer do curso além de concluir seus componentes curriculares obrigatórios. Este surgiu devido à pouca demanda de empresas de Tecnologia da Informação (TI) na região onde a universidade se encontra, em Quixadá no Ceará.

Dentro do NPI, os projetos desenvolvidos têm como objetivo construir soluções que facilitem as atividades do cotidiano da universidade, esta que tem um grande interesse no desenvolvimento destes projetos, pois consegue reduzir custos ao priorizar construções de sistemas internamente \cite{npi2013}. Em paralelo, aumenta a qualidade dos profissionais formados por ela, além de proporcionar um ambiente real de trabalho que facilita a entrada dos concludentes no mercado de trabalho.

No NPI existe um modelo de processo definido em que os desenvolvedores devem seguir para o exercício de suas atividades \cite{npi2013}. Entretanto, este não é devidamente seguido, ocasionando uma despadronização na maneira como estes  desenvolvedores trabalham em seus projetos. Somado-se a isto, o NPI apresenta problemas tais como: "Baixa qualidade da documentação dos sistemas; [\ldots] Falta de uma equipe de manutenção; [\ldots] Rotatividade dos profissionais" o que acaba gerando graves problemas de manutenção \cite[p.~4]{paduelli2006}.


Segundo \citeonline[p.~170]{sommerville2011}, "a manutenção de software é o processo geral de mudança em um sistema depois que ele é liberado para uso". Sendo assim, entende-se como manutenção de software  qualquer alteração realizada no sistema após este ser considerado "pronto" e implantado em seu ambiente de operação. Atualmente, esta vem ganhando uma maior atenção por parte das empresas desenvolvedoras de software. Isso acontece devido aos altos custos na fase de manutenção, podendo atingir 70\% do esforço total aplicado no projeto, além de sofrer possíveis aumentos ao longo da produção do software \cite{pressman2010}.

Mudanças no software são inevitáveis, e não possuem regra, simplesmente acontecem. Garantir que essas mudanças sejam devidamente controladas, identificadas é o papel da  Gerência de Configuração (GC), esta que tem sua importância evidenciada quando diferentes modelos de maturidade o abordam como MPS.BR no nível F e o CMMI(Capability Maturity Model Integration) \cite{furlaneto2006}. Para identificar e auxiliar no controle das mudanças, um conjunto de ferramentas CASE (Computer-Aided Software Engineering) são utilizadas pela Gerência de Configuração.

A experiência em aplicar ferramentas  de gestão de configuração com o objetivo de melhorar a manutenibilidade de uma fábrica de software é explanado por \citeonline{poliana2005}. Eles inseriram a utilização destas ferramentas para evitar a inserção de novos erros proveniente de manutenções realizadas no software. Diferentemente do trabalho a ser desenvolvido nesse projeto que busca verificar o impacto de uma ferramenta em específico, de Integração Contínua,  na melhoria da manutenibilidade.

Entende-se Integração Contínua como uma ferramenta de gestão de configuração, que auxilia os desenvolvedores e permite que as mudanças realizadas no software sejam imediatamente avaliadas, testadas, verificadas de modo a prover um  \textit{feedback} imediato para correção de possíveis erros de integração que somente seriam verificados futuramente após problemas mais complexos de integração \cite{paul2007}.

A justificativa pela escolha do tema se deu através da ausência de pesquisas que busquem avaliar o impacto que o uso de uma ferramenta de integração contínua utilizada durante desenvolvimento de um produto de software exerce sobre a manutenibilidade do software produzido \cite{rodrigo2007}. As buscas por pesquisas foram realizadas no: acervo da Sociedade Brasileira de Computação (SBC), \textit{Google Academys} utilizando os termos: \textit{Continuous Integration, Continuous Integration and maintainability index}. Proporcionar conhecimento para o mercado de modo a ajudar empresas a avaliarem a necessidade, viabilidade do uso de ferramentas deste gênero.