\chapter*[Introdução]{Introdução}
\addcontentsline{toc}{chapter}{Introdução}
% ----------------------------------------------------------

O presente trabalho visa aplicar no Núcleo de Práticas em Informática (NPI) a utilização de uma ferramenta de Integração Contínua (IC), buscando melhorar as taxas de manutenibilidade dos sistemas de software produzidos e avaliar a aceitação e facilidade por parte dos desenvolvedores acerca da ferramenta supracitada.

O NPI é o local onde estudantes que estão em ano de conclusão de curso podem estagiar e aprimorarem seus conhecimentos adquiridos no decorrer do curso e concluir seus componentes curriculares obrigatórios. O NPI surgiu por conta da pouca demanda de empresas de Tecnologia da Informação (TI) na região onde a universidade se encontra, em Quixadá no Ceará. Dentro do NPI os projetos desenvolvidos têm como objetivo construir soluções que facilitem as atividades do cotidiano da universidade, esta que tem um grande interesse no desenvolvimento destes projetos, pois consegue reduzir custos ao priorizar construções de sistemas internamente, e em paralelo aumentar a qualidade dos profissionais formados por ela, além de proporcionar um ambiente real de trabalho que facilita a entrada dos concludentes no mercado de trabalho. \cite{npi2013}

Segundo \citeonline[p.~170]{sommerville2011} "a manutenção de software é o processo geral de mudança em um sistema depois que ele é liberado para uso". Sendo assim entende-se como manutenção de software  qualquer alteração realizada no sistema após este ser considerado "pronto" e implantado em seu ambiente de operação.

Atualmente, a manutenção de software vem ganhando uma maior atenção por parte das empresas desenvolvedoras de software. Isso acontece devido a maior parte dos gastos de um produto de software estarem na fase de manutenção\apud{pigoski1997}{figueiredo2005}\apud{polo2002}{figueiredo2005}. Custos esses que segundo \citeonline{pressman2010} equivalem a  cerca de 70\% do esforço aplicado.

Entende-se Integração Contínua como uma ferramenta de auxílio aos desenvolvedores que permite que as mudanças realizadas no software sejam imediatamente avaliadas,testadas,verificadas de modo à prover um  \textit{feedback} imediato para correção de possíveis erros de integração que somente seriam verificados futuramente após problemas mais complexos de integração.\cite{paul2007}

A experiência em aplicar ferramentas  de gestão de configuração com o objetivo de melhorar a manutenibilidade de uma fábrica de software é abordada através de um conjunto de ferramentas automatizadas que buscam melhorar a manutenibilidade do software \cite{poliana2005}. Diferentemente do trabalho proposto que busca verificar o impacto de uma ferramentas em específico de Integração Contínua  para a melhoria da manutenibilidade.

\cite{furlaneto2006} relata a construção de uma ferramenta, um modelo de processo, que apoiasse a gerência configuração, ferramentas foram avaliadas e comparadas com a ferramenta desenvolvida, verificando assim, que a ferramenta desenvolvida atendeu bem o escopo definido pelo MPS.BR (Melhoria de Processo Brasileiro)  e após ser comparada com ferramentas existentes, esta conseguiu atender bem os requisitos definidos. O trabalho citado difere do proposto ao criar uma ferramenta que apoie a GC, enquanto o proposto visa avaliar o impacto de uma ferramenta específica em projetos e pessoas.

Mudanças no software são inevitáveis, elas não possuem regra, simplesmente acontecem. Garantir que essas mudanças sejam devidamente controladas, identificadas é o papel da  Gerência de Configuração (GC). A importância da GC fica evidenciada quando diferentes modelos de maturidade o abordam como o MPS.BR no nível F e o CMMI(Capability Maturity Model Integration) \cite{furlaneto2006}.

A justificativa pela escolha do tema se dá pela ausência de pesquisas que tenham como objetivo avaliar o impacto que uma ferramenta de integração contínua exerce sobre a manutenibilidade de um produto de software. Gerar conhecimento para o mercado de modo que ajude empresas a avaliar a necessidade e a viabilidade do uso de ferramentas deste tipo nas empresas.