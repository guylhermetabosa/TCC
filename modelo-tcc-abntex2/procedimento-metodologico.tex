\chapter{Procedimentos Metodológicos}\label{metodologia}

\begin{itemize}
\item Identificar e analisar o desenvolvimento das atividades no NPI.
\end{itemize}
A identificação e análise das atividades do NPI serão realizadas por meio de entrevistas com líderes técnicos de equipes, professores supervisores e avaliação do modelo de processo proposto por \citeonline{npi2013} com ênfase na atividade de Gerência de Configuração.

\begin{itemize}
\item Definir a utilização de uma ferramenta de integração contínua.
\end{itemize}
A ferramenta utilizada será a CruiseControl \footnote{http://cruisecontrol.sourceforge.net/} por ser uma ferramenta \textit{open source} e por fornecer suporte a diferentes ferramentas de build.

\begin{itemize}
\item Avaliar o nível de manutenibilidade do software anteriormente e posteriormente a utilização da ferramenta de integração contínua.
\end{itemize}
Para avaliação da manutenibilidade so software será utilizado a medida do Índice de Manutenibilidade (Maintainability Index - MI). Para obtenção do MI será preciso a utilização de algumas medidas como o Volume de Halstead e a quantidade de linhas de código. Uma fórmula para o cálculo da manutenibilidade é fornecido pelo trabalho de \citeonline{gustavo2013}.
\begin{itemize}

\item Implementar a utilização da ferramenta e coletar resultados.
\end{itemize}
O processo de implementação será por meio de treinamento para explanação da importância da ferramenta seus benefícios e entendimento de sua funcionalidade. Ao final um relatório será gerado para análise dos dados.





\section{Cronograma de Execução}
\begin{figure}[tbh]
\centering
\includegraphics[width=0.9\linewidth]{./images/cronograma}
\caption[Cronograma de Execução]{Cronograma de Execução}
\label{fig:Cronograma}
\end{figure}