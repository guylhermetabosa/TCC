\chapter{Procedimentos Metodológicos}\label{metodologia}

\section{Analisar as atividades do Núcleo de Práticas}
Esta atividade visa identificar como as atividades ocorriam dentro do Núcleo de Práticas em Informática, para tal foi analisado o processo\footnote{http://www.npi.quixada.ufc.br/processo/} existente modelado pela ferramenta \textit{EPF Composer}. Além da experiência do autor pois este era um funcionário da organização com experiência de 8 meses nas atividades lá realizadas. 

\section{Pesquisar e selecionar a ferramenta de Integração Contínua}

Esta atividade consiste em  colher informações e selecionar a ferramenta que melhor se adapta a realidade existente no Núcleo de Práticas em Informática. 
Para a escolha da ferramenta foi preciso definir um conjunto de requisitos que a ferramenta deveria possuir e suprir, de modo a filtrar a ferramenta escolhida dentre as diversas existentes. As definições foram descritas \autoref{escolhaFerramenta}. 

\section{Absorção do perfil dos estagiários do Núcleo de Práticas}
Esta atividade tem como objetivo entender e identificar os conhecimentos dos estagiários do núcleo acerca de integração contínua, experiência de uso em projetos pessoais, conhecimento na ferramenta e como esta funciona.

\section{Implantação da ferramenta de integração contínua}
A implantação será realizada no núcleo e a utilização da ferramenta em um projeto piloto.


\pagebreak
\section{Cronograma de Execução}

\begin{table}[!htpb]
\centering
\caption{Cronograma das atividades previstas}

% definindo o tamanho da fonte para small
\begin{small} 
  
% redefinindo o espaçamento das colunas
\setlength{\tabcolsep}{6pt} 

% \cline é semelhante ao \hline, porém é possível indicar as colunas que terão essa a linha horizontal
% \multicolumn{10}{c|}{Meses} indica que dez colunas serão mescladas e a palavra Meses estará centralizada dentro delas.

\begin{tabular}{|c|c|c|c|c|c|c|c|c}\hline
 & \multicolumn{7}{c|}{2014}\\ \cline{2-8}
\raisebox{1.5ex}{ATIVIDADES} & Mai & Jun & Jul & Ago & Set & Out & Nov \\ \hline

Estudo de Campo & X & X & & &  &  & \\ \hline
Defesa do Projeto &  & X & & &  &  & \\ \hline
Avaliação do Processo do NPI &  &  & X & X &  &  & \\ \hline
Avaliação da Manutenibilidade do GAL &  &  &  & X & X &  & \\ \hline
Implantação da Integração Contínua &  &  &  & X & X &  & \\ \hline
Treinamento do uso da Ferramenta &  &  &  & & X & X & \\ \hline
Avaliação da Manutenibilidade do GAL  &  &  &  &  &  & X & \\ \hline
Análise do Dados de Manutenibilidade &  &  &  & &  & X &  \\ \hline
Revisão final da monografia & & & & & & X & X \\ \hline
Defesa do Projeto & & & & & & & X \\ \hline

\end{tabular} 
\end{small}
\legend {Fonte: Elaborado pelo Autor}
\label{t_cronograma}
\end{table} 