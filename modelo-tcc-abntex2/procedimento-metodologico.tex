\chapter{Procedimentos Metodológicos}\label{metodologia}

Esta seção tem como objetivo descrever os passos tomados que irão permitir concluir os objetivos definidos anteriormente.

O presente trabalho possui as seguintes características:

\begin{itemize}
\item \textbf{Pesquisa Aplicada}: Demonstra a utilização de métricas de software sobre o objeto em estudo;

\item \textbf{Pesquisa Explicativa}: Explicita práticas utilizadas em um ambiente de integração contínua e analisa como estas práticas influenciam a manutenibilidade do sotware;

\item \textbf{Pesquisa Quanti-Qualitativa}: Caracteriza-se como quantitiva pois traduz em números as informações obtidas para serem analisadas, mais precisamente a taxa de manutenibilidade. Qualitativa pois visa melhorar um atributo de qualidade, a manutenibilidade.
\end{itemize}


\begin{itemize}
\item Identificar e analisar o desenvolvimento das atividades no NPI.
\end{itemize}
A identificação e análise das atividades do NPI serão realizadas por meio de entrevistas com líderes técnicos de equipes, professores supervisores e avaliação do modelo de processo proposto por \citeonline{npi2013} com ênfase na atividade de Gerência de Configuração.

\begin{itemize}
\item Definir a utilização de uma ferramenta de integração contínua.
\end{itemize}
A ferramenta utilizada será o Jenkins \footnote{http://jenkins-ci.org/} por ser uma ferramenta \textit{open source} e por fornecer suporte a diferentes ferramentas de build.

\begin{itemize}
\item Avaliar o nível de manutenibilidade do software anteriormente e posteriormente a utilização da ferramenta de integração contínua.
\end{itemize}
Para avaliação da manutenibilidade so software será utilizado a medida do Índice de Manutenibilidade (Maintainability Index - MI). Para obtenção do MI será preciso a utilização de algumas medidas como o Volume de Halstead e a quantidade de linhas de código. Uma fórmula para o cálculo da manutenibilidade é fornecido pelo trabalho de \citeonline{isaias2012}.
\begin{itemize}

\item Implementar a utilização da ferramenta e coletar resultados.
\end{itemize}
O processo de implementação será por meio de treinamento para explanação da importância da ferramenta seus benefícios e entendimento de sua funcionalidade. Ao final um relatório será gerado para análise dos dados. Os dados serão obtidos através da aplicação da fórmula de Manutenibilidade previamente definida antes e após a implantação da ferramenta de integração contínua, e os dados obtidos por meio da fórmula, serão confrontados a fim de fornecer os resultados desejados.
\pagebreak
\section{Cronograma de Execução}

\begin{table}[!htpb]
\centering
\caption{Cronograma das atividades previstas}

% definindo o tamanho da fonte para small
\begin{small} 
  
% redefinindo o espaçamento das colunas
\setlength{\tabcolsep}{6pt} 

% \cline é semelhante ao \hline, porém é possível indicar as colunas que terão essa a linha horizontal
% \multicolumn{10}{c|}{Meses} indica que dez colunas serão mescladas e a palavra Meses estará centralizada dentro delas.

\begin{tabular}{|c|c|c|c|c|c|c|c|c}\hline
 & \multicolumn{7}{c|}{2014}\\ \cline{2-8}
\raisebox{1.5ex}{ATIVIDADES} & Mai & Jun & Jul & Ago & Set & Out & Nov \\ \hline

Estudo de Campo & X & X & & &  &  & \\ \hline
Defesa do Projeto &  & X & & &  &  & \\ \hline
Avaliação do Processo do NPI &  &  & X & X &  &  & \\ \hline
Avaliação da Manutenibilidade do GAL &  &  &  & X & X &  & \\ \hline
Implantação da Integração Contínua &  &  &  & X & X &  & \\ \hline
Treinamento do uso da Ferramenta &  &  &  & & X & X & \\ \hline
Avaliação da Manutenibilidade do GAL  &  &  &  &  &  & X & \\ \hline
Análise do Dados de Manutenibilidade &  &  &  & &  & X &  \\ \hline
Revisão final da monografia & & & & & & X & X \\ \hline
Defesa do Projeto & & & & & & & X \\ \hline

\end{tabular} 
\end{small}
\legend {Fonte: Elaborado pelo Autor}
\label{t_cronograma}
\end{table} 