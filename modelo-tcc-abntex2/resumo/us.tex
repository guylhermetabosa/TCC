% resumo em inglês
\begin{resumo}[Abstract]
 \begin{otherlanguage*}{english}
The agile development is day by day inserted on the quotidian of software development companies. The growing seek for agility in the development and market competitiveness has impacted in the existence of this scenario. Therefore, many companies seek to apply the defined methodologies in this kind of development	in their processes, however, this is not an easy task. The definition and deployment of this practices executed in a way not adequate enough can bring contrary results to the expected. So, this work had as an aim to implant the using of a continuous integration tool on a software factory, the Informatics Practice Center (NPI) at UFC in Quixadá. A continuous integration tool is inserted in one of the practices defined by Extreme Programming (XP). In order to reach this goals, the current process of NPI was analyzed, beyond the author experience thas was a member of this software factory. Added to this stage was defined the continuous integration tool to be implanted according with a set of specified characteristics, the knowledges of the factory's members about Configuration Management and Continuous Integration was evaluated, and finally the deployment, with the definition of positives and negatives points found at the deployment.

   \vspace{\onelineskip}
 
   \noindent 
   \textbf{Key-words}: Continuous Integration. Agile Development. Configuration Management.
 \end{otherlanguage*}
\end{resumo}