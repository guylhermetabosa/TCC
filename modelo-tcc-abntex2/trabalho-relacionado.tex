\chapter{Trabalhos Relacionados}\label{trabalhorel}
Como citado anteriormente a atividade de manutenção demanda altos custos para os produtos de software. Definir técnicas e abordagens para minimizar estes custos são de fundamental importância. \citeonline{poliana2005} afirmam que durante a fase de manutenção existe a chance de novos erros serem inseridos no software, para evitar que tais erros sejam inseridos um conjuntos de ferramentas de gestão de configuração foram utilizadas. Um abordagem que visa que erros sejam inseridos é explanada por \citeonline{gleiph2011} por meio do \textit{Ouriço}, que busca garantir a consistência dos repositórios de trabalho, através de verificações semânticas e/ou sintáticas.	Além de erros serem inseridos, outros problemas ocorrem durante a fase de manutenção um deles é a falta de documentação encontrada pela equipe de manutenção. \citeonline{sergio2005} mostrou a importância de cada artefato para a manutenção do software, onde os artefatos que foram considerados menos importantes obtiveram um índice de indicação de 50\%, e de todos os artefatos avaliados, 58\% foram utilizados pelos mantenedores.
