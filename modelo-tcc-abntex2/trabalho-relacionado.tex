\chapter{Trabalhos Relacionados}\label{trabalhorel}
Nesta seção será descrito trabalhos que influenciaram os conceitos envolvidos neste trabalho além de demonstrar pontos comuns e distintos entre si e o proposto.

O trabalho de \citeonline{pereira2013} descreve a implantação de uma ferramenta de integração contínua em um departamento de desenvolvimento e pesquisa, o Sedna, de uma empresa de engenharia em um ambiente de MPS.BR nível F. Os principais clientes do Sedna são voltados a área de óleo e gás.

Durante o período de implantação, o Sedna fornecia manutenção a três sistemas, onde um tratava-se de uma aplicação web, deste modo o \textit{deploy} da aplicação para todos os seus usuários era de responsabilidade do Sedna. Tal ação tornava-se bastante custosa devido ao grande número de web sites que deveriam ser atualizados.
	
Dentro do Sedna algumas ferramentas de gerência de configuração já eram utilizadas, tais como o Atlassian Jira, Subversion (SVN) e o Atlassian Confluence. Ainda com a utilização destas ferramentas a equipe possuia grandes dificuldades no tempo de realização do \textit{deploy}, pois esta atividade consumia uma grande parte do tempo da equipe, tempo de aprendizagem e realização do \textit{deploy} da aplicação principalmente para novos membros da equipe e um \textit{feedback} atrasado para problemas básicos de commits errôneos.

Os autores descrevem que como pontos positivos da integração estão a utilização de uma ferramenta de integração contínua da mesma empresa que fornecia o sistema de gerenciamento de projetos, a ferramenta utilizada foi o Atlassian Bamboo e a utilização de ferramentas da automação do processo de build o que facilitava o trabalho da integração contínua. Bem como o autor destaca as experiências negativas da implantação, que estão na ausência de uma máquina com requisitos mínimos exigidos para a utilização, a ausência de treinamento da equipe, onde os conhecimentos de IC estava com o líder de projeto e o gerente de configuração e por fim a ferramenta não fornecia suporte ao \textit{redeploy} dos artefatos gerados, o que criou um barreira na equipe acerca da ferramenta.


\cite{abdul2012}

Uma das principiais vantagens da utilização da integração contínua é o seu \textit{feedback} imediato acerca de problemas de integração, e entender e interpretar as principais causas dos problemas de integração foi realizado por \citeonline{miller2008}. Quando este percebeu que em um projeto da Microsoft, o Service Factory, a maior causa de quebra de build eram violações no sistema de análise de código, seguido por testes automatizados e erros de compilação.

A coleta de métricas através da inspeção contínua é uma grande vantagem incluída dentro da Integração Contínua, pois a sua principal vantagem é aumentar a qualidade do produto e facilitar o processo de manutenção do software. Para isso \citeonline{moreira2010} desenvolveram um framework para extração de métricas de forma automatizada implementado em um ambiente de Integração Contínua, logo ficou evidente importância das inspeções e teste de códigos nas builds como foi descrito anteriormente.

Os testes por serem peça fundamental na qualidade no software e na confiança do produto, além de serem fundamentais em uma build de integração contínua, \citeonline{kim2009} propuseram a criação de um framework automatizado de testes que permitissem facilitar os casos de testes, reuso de componentes, relatórios mais legíveis e integrado com um ambiente de integração contínua.
