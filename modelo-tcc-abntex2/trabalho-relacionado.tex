\chapter{Trabalhos Relacionados}\label{trabalhorel}
Nesta seção será descrito trabalhos que influenciaram os conceitos envolvidos neste trabalho além de demonstrar pontos comuns e distintos entre si e o proposto. A \autoref{trabrelisaias} descreve o trabalho de \citeonline{isaias2012} e a \autoref{trabrelgustavo} descreve o trabalho de \citeonline{gustavo2013}.
\section{Impacto de Padrões de Projeto na Manutenibilidade} \label{trabrelisaias}
Analisar atributos de qualidade de software faz se importante para quantificar a qualidade de um produto, esta quantificação é obtida por meio de medição. \citeonline{isaias2012} buscou avaliar a taxa de manutenibilidade de um produto de software, através da utilização de métricas específicas para mensurar a manutenibilidade de um software. Com o objetivo de avaliar o impacto da utilização de um conjunto de boas práticas de programação, o autor mensurou o sistema a ser estudado, refatorou utilizando tais práticas, e posteriormente refez as medições.

O trabalho comprovou que a utilização de boas práticas de programação foi eficaz na melhora das taxas de manutenibilidade, que por consequência melhora a qualidade do software.

O trabalho supracitado difere do proposto pois, este refatorou um sistema existente para fins de análise da utilização das boas práticas, entretanto o trabalho a ser desenvolvido não tem como intuito refatorar um sistema legado, mas verificar o impacto que a utilização de um ferramenta de integração contínua exercerá na manutenibilidade de um software ainda em construção, esta sendo inserida durante o desenvolvimento do mesmo, e por fim implantar a utilização deste tipo de ferramenta.


 
\section{Manutenibilidade em Softwares Advindos de Linhas de Produto de Software} \label{trabrelgustavo}
Avaliar a manutenibilidade de software como citado anteriormente é importante para analisar a qualidade de um produto de software de modo quantitativo, o trabalho de \citeonline{gustavo2013} busca identificar o nível de manutenibilidade de software derivados de linhas de produtos de software, o que é importante pois este tipo de software é característico pela reuso de componentes, conhecido como \textit{features}, em larga escala. O referido trabalho definiu sete critérios para melhoria de manutenibilidade de software sob seis produtos de software derivados de uma LPS (Linha de Produtos de Software) definida como TankWar, que é um jogo desenvolvido na Alemanha, na universidade de Magdeburg. Os sete critérios definidos foram: Refinamento, Responsabilidades, Clones de Código, Métodos novos e sobrescritos, Responsabilidades em características, Encapsulamento, Convenção de Código. 

O trabalho definia diretrizes, que quando um critério não era devidamente atendido, este era atendido pela diretriz, que tinha como objetivo aumentar o índice de manutenibilidade. Assim concluiu-se que a definição das diretriz foi eficiente na melhoria da manutenibilidade de software advindos de linhas de produto de software \cite{gustavo2013}.
