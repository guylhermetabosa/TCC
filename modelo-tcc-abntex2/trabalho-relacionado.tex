\chapter{Trabalhos Relacionados}\label{trabalhorel}
Nesta seção será descrito trabalhos que influenciaram os conceitos envolvidos neste trabalho além de demonstrar pontos comuns e distintos entre si e o proposto. A \autoref{trabrelisaias} descreve o trabalho de \citeonline{isaias2012} e a \autoref{trabrelgustavo} descreve o trabalho de \citeonline{gustavo2013}.
\section{Impacto de Padrões de Projeto na Manutenibilidade} \label{trabrelisaias}
Analisar atributos de qualidade de software faz se importante para quantificar a qualidade de um produto, esta quantificação é obtida por meio de medição. \citeonline{isaias2012} buscou avaliar a taxa de manutenibilidade de um produto de software, através da utilização de métricas específicas para mensurar a manutenibilidade de um software. Com o objetivo de avaliar o impacto da utilização de um conjunto de boas práticas de programação, o autor mensurou o sistema a ser estudado, refatorou utilizando tais práticas, e posteriormente refez as medições.

O trabalho comprovou que a utilização de boas práticas de programação foi eficaz na melhora das taxas de manutenibilidade, que por consequência melhora a qualidade do software.

O trabalho supracitado difere do proposto pois, este refatorou um sistema existente para fins de análise da utilização das boas práticas, entretanto o trabalho a ser desenvolvido não tem como intuito refatorar um sistema legado, mas verificar o impacto que a utilização de um ferramenta de integração contínua exercerá na manutenibilidade de um software ainda em construção, esta sendo inserida durante o desenvolvimento do mesmo, e por fim implantar a utilização deste tipo de ferramenta.


 
\section{Manutenibilidade em Softwares Advindos de Linhas de Produto de Software} \label{trabrelgustavo}
