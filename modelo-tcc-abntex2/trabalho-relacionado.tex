
\chapter{Trabalhos Relacionados}\label{trabalhorel}

\citeonline{poliana2005} relatam, o uso de ferramentas de gerência de configuração e \textit{frameworks} de teste unitário em uma fábrica de software. Para isso foram utilizadas cinco ferramentas automatizadas. Primeiramente foi analisado o cenário atual da fábrica de software, e proposto um novo modelo. Dois projetos da empresas desenvolvidos em C++ e ambiente linux foram utilizados neste trabalho. Onde um projeto que estava sendo mantido possuía aproximadamente 140 mil linhas de código e o outro projeto possuía aproximadamente 675 mil linha de código. Após a integração das ferramentas, no cenário antigo os desenvolvedores somente testavam os casos de teste que eles achavam haviam sido afetado pelas mudanças realizadas, no cenário proposto todo os casos de teste eram reexecutados, o resultados obtidos mostraram que para o primeiro projeto apenas 40\% dos casos de testes tiveram sucesso, e no outro projeto 8\%, após o uso da ferramenta os número aumentaram para 70 \% para o primeiro projeto e 95\% para o segundo. Além disso foi comparado o tempo que a build fica quebrada antes e depois do uso das ferramentas, verificou-se que antes da introdução da ferramenta a build  ficava uma média de 2,55 dias quebrada e depois da implantação da ferramenta esta passou a ficar em média 1 dia quebrada, já o outro projeto não foi possível verificar os dados pois este não havia build. Outro ponto observado foram as quantidades e tamanho das integrações antes da introdução da ferramentas.As integrações eram feitas apenas uma vez ao dia, e com muitas alterações, antes da implantação da ferramenta a média de linhas alteradas estavam em cerca de 170 linhas por alteração enquanto depois do uso da ferramenta passou a ser 66 em média.