\chapter{Trabalhos Relacionados}\label{trabalhorel}
Nesta seção será descrito trabalhos que influenciaram os conceitos envolvidos neste trabalho além de demonstrar pontos comuns e distintos entre si e o proposto.

O trabalho de \citeonline{pereira2013} descreve a implantação de uma ferramenta de integração contínua em um departamento de desenvolvimento e pesquisa o Sedna, de uma empresa de engenharia em um ambiente de MPS.BR nível F. Os principais clientes do Sedna são voltados a área de óleo e gás.

Durante o período de implantação, o Sedna fornecia manutenção a três sistemas, onde um  deles era uma aplicação web, deste modo o \textit{deploy} da aplicação para todos os seus usuários era de responsabilidade do Sedna. O que tornava-se bastante custoso devido ao grande número de web sites que deveriam ser atualizados.

Dentro do Sedna algumas ferramentas de gerência de configuração já eram utilizadas, tais como o Atlassian Jira, Subversion (SVN) e o Atlassian Confluence. Ainda com a utilização destas ferramentas a equipe possui grandes dificuldades no tempo de realização do \textit{deploy}, pois esta atividade consumia uma grande parte do tempo da equipe, aprender a realizar o \textit{deploy} da aplicação, principalmente para novos membros da equipe e um \textit{feedback} atrasado para problemas básicos de commits com erros.

Os autores descrevem que como pontos positivos da integração estão a utilização de uma ferramenta de integração contínua da mesma empresa que fornecia o sistema de gerenciamento de projetos, a ferramenta utilizadas foi o Atlassian Bamboo e a utilização de ferramentas da automação do processo de build o que facilitava o trabalho da integração contínua. Bem como o autor destaca as experiências negativas da implantação, que estão na ausência de uma máquina com requisitos mínimos exigidos para a utilização, a ausência de treinamento da equipe, onde os conhecimentos de IC estava com o líder de projeto e o gerente de configuração e por fim a ferramenta não fornecia suporte ao \textit{redeploy} dos artefatos gerados, o que criou um barreira na equipe acerca da ferramenta.

\cite{moreira2010}